% =========================
% Paquetes básicos
% =========================
\usepackage{amsmath,amssymb}      % símbolos y entornos matemáticos
\usepackage{fontspec}             % fuentes OpenType/TrueType (XeLaTeX/LuaLaTeX)
\usepackage{etoolbox}             % centrar automáticamente todas las figuras creadas por pandoc

% =========================
% Fuentes del documento
% =========================
\setmainfont{FiraCode Nerd Font}       % fuente principal (texto)
\setmonofont{FiraCode Nerd Font Mono}  % fuente monoespaciada (código)

% =========================
% Microtipografía
% =========================
\usepackage{microtype}            % mejora kerning/expansión para mejor legibilidad

% =========================
% Gestión de imágenes
% =========================
\usepackage{graphicx}             % \includegraphics
% escala global: cada imagen ocupa 80% del ancho del texto, respetando proporción
\setkeys{Gin}{width=0.80\linewidth,keepaspectratio}

% =========================
% Control de flotantes (figuras/tablas)
% =========================
\usepackage{float}                % opción [H] para fijar posición exacta
\makeatletter
\def\fps@figure{H}                % por defecto, colocar figuras en la posición indicada
\makeatother
\usepackage[section]{placeins}    % \FloatBarrier automático al final de cada sección

% =========================
% Hipervínculos
% =========================
\hypersetup{colorlinks=true, linkcolor=blue, urlcolor=blue}

% =========================
% Interlineado
% =========================
\usepackage{setspace}
\setstretch{1.5}                  % interlineado 1.5

% =========================
% Tablas compactas (estilo y tamaño)
% =========================
% macro de estilo: tamaño, espacio entre columnas y entre filas
\newcommand{\TableTight}{%
  \footnotesize                % tamaño del texto de la tabla
  \setlength{\tabcolsep}{5pt}  % separación entre columnas
  \renewcommand{\arraystretch}{1.0}% altura de filas (1=normal)
}

% aplicar estilo automáticamente a tablas
\AtBeginEnvironment{table}{\TableTight}
\AtBeginEnvironment{longtable}{\TableTight}
\AtBeginEnvironment{tabular}{\footnotesize}
\AtBeginEnvironment{figure}{\centering}

% =========================
% Comando de Pandoc (referencia de compilación)
% =========================
% pandoc --from=markdown+implicit_figures contexto.md -o contexto.pdf --pdf-engine=xelatex -V geometry:top=0.67in -V geometry:bottom=0.67in -V geometry:left=0.85in -V geometry:right=0.85in -H header.tex --resource-path=.:images:../images
